\input{text/diss}
\usepackage{setspace}

\begin{document}

\def\labauthors{Понур К.А., Сарафанов Ф.Г., Сидоров Д.А.}
\def\labgroup{420}
\def\labnumber{3000}
\def\labtheme{Гармонический анализ и синтез периодических сигналов}
\renewcommand{\vec}{\mathbf}
\renewcommand{\Re}{\operatorname{Re}}
\renewcommand{\Im}{\operatorname{Im}}
\renewcommand{\phi}{\varphi}
\renewcommand{\kappa}{\varkappa}
\renewcommand{\hat}{\widehat}
%%%%%%%%%%%%%%%%%%%%%%%%%%%%%%%%%%%%%%%%%%%%%%%%%%%%%%%%%%%%%%%%%%%%%%%%%%%%%%%
\input{text/titlepage}
%%%%%%%%%%%%%%%%%%%%%%%%%%%%%%%%%%%%%%%%%%%%%%%%%%%%%%%%%%%%%%%%%%%%%%%%%%%%%%%
\begin{spacing}{1}
\tableofcontents
\end{spacing}
% \setstretch{1.2}
\newpage
%%%%%%%%%%%%%%%%%%%%%%%%%%%%%%%%%%%%%%%%%%%%%%%%%%%%%%%%%%%%%%%%%%%%%%%%%%%%%%%
\section{Введение}
 	Изменяющийся во времени сигнал $S(t)$ называется периодическим, если для него выполняется условие
 	\begin{equation}
 		S(t)=S(t+kT),
 	\end{equation}
 	где $T$-- период изменения, а $k$-- любое целое число
Если периодическая функция $S(t)$ удовлетворяет условиям Дирихле, то есть является огранченной и имеет в пределах одного периода конечное число экстремумов и разрывов, то согласно теореме Фурье онра может быть представлена  виде тригонометричекого ряда, называемого рядом Фурье
\begin{equation}
	\label{eq:1}
	S(t)=\frac{a_0}{2}+\sum_{n=1}^\infty(a_n\cos(n\Omega )+b_n\sin{n\Omega t})
\end{equation}
коэффициенты которго определяются из выражений
\begin{equation}
	\label{eq:2.1}
	a_n=\frac{2}{T}\int\limits_{-\frac{T}{2}}^{\frac{T}{2}}S(t)\cos(n\Omega t )dt
\end{equation}
\begin{equation}
	\label{eq:2.2}
	b_n=\frac{2}{T}\int\limits_{-\frac{T}{2}}^{\frac{T}{2}}S(t)\sin(n\Omega t )dt,
\end{equation}
а угловая частота $\Omega$ связана с периодом $T$ соотношением:
\begin{equation}
	\Omega=\frac{2\pi}{T}
\end{equation}
Пользуясь формулами тригонометрии ряд (\ref{eq:1}) можно записать в виде
\begin{equation}
\label{eq:3}
 	S(t)=\frac{a_0}{2}+\sum_{n=1}^{\infty}A_n\cos(n\Omega t-\theta_n)
 \end{equation} 
 более наглядно определяющем совокупность гармонических составляющих, на которые раскладывается исходная функция $S(t)$. Такая совокупность называется спектром. Для периодических функций спектр является дискретным и состоит из постоянной составляющей, которую можно рассматривать как гарминику с нулевой частотой и амплитудой $A_0=a_0/2$ и бесконечного множества гармонических составляющих с частотами $\omega_n=n\Omega$, кратными основной частоте $\Omega$, амплитудами $A_n$ и начальными фазами $\theta_n$.

 Для наглядности спектры удобно изображать в виде спектральных диаграмм -- амплитудных и фазовых.
 Используя формулу
 \begin{equation}
 	\cos\alpha=\frac{e^{i\alpha}+e^{-i\alpha}}{2}
 \end{equation}
 можно от (\ref{eq:3}) перейти к комплексной форму ряда Фурье
 \begin{equation}
 	\label{eq:4}
 	S(t)=\frac{a_0}{2}+\Re\sum_{n=1}^{\infty}A_ne^{in\Omega t}=\frac{1}{2}\sum_{n=1}^{\infty}A_ne^{in\Omega t}, %как пишется крышечка??
 \end{equation}
 где комплексная амплитуда $A_n=A_ne^{\pi\theta_n}$ содержит информацию об комплексной амплитуде и фазе n-ой гармоники. Сравнивая ряды (\ref{eq:4}) и (\ref{eq:1}) и используя соотношения (\ref{eq:2.1}) и (\ref{eq:2.2}), можно получить формулу для вычисления комплесных амплитуд
 \begin{equation}
 	A_n=\frac{2}{T}\int\limits_{-\frac{2}{T}}^{\frac{T}{2}}
 \end{equation}

 Отметим некоторые свойства рядов Фурье.

 1. Если $S(t)$- четная функция, то в выражении (\ref{eq:1}) $b_n$=0, а в разложении остаются только косинусоидальные члены.

 2. Для нечетной функции $S(t)$ в разложении (\ref{eq:1}), наоборот, лишь коэффициенты $b_n$ отличны от нуля, а в $a_n=0$;

 3. При увеличении периода сигнала $T$ расстояние по оси частот между соседними спектральными компонентами уменьшается, то есть спектр становится более <<плотным>>;
 4. Для импульсных сигналов, у которых промежуток времени, когда сигнал отличен от нуля, неизменен, при увеличении периода величина Фурье-компонент уменьшаетсмя обратно пропорционально T.

\newpage
\section{Расчет, синтез и экспериментальное получение различных сигналов}
\subsection{Разложения в ряд Фурье}
\subsubsection{Треугольник}
Функция имеет вид 
\begin{equation}
	U(t)=\left\{
	\begin{aligned}
		U_0+\frac{4U_0}{T}t,\quad &t\in [-\frac{T}{2},0]\\
		U_0-\frac{4U_0}{T}t,\quad &t\in[0, \frac{T}{2}]
	\end{aligned}
	\right.
\end{equation}
\begin{gather}
	a_n=\frac{2}{T}\left[\int\limits^0_{-\frac{T}{2}}(U_0+\frac{4U_0}{T}t)\cos(n\Omega t)dt+\int\limits_0^{\frac{T}{2}}(U_0-\frac{4U_0}{T}t)\cos(n\Omega t)dt \right]=\\=
	-\frac{U_0 (\pi  n \sin (\pi  n)+2 \cos (\pi  n)-2)}{\pi ^2 n^2}-\frac{U_0 (\pi  n \sin (\pi  n)+2 \cos (\pi  n)-2)}{\pi ^2 n^2}=\\=
	\frac{4U_0}{\pi^2 n^2}(1-\cos{\pi n})
	% 
	% =\\=\frac{2}{T}\left[\frac{U_0}{n\Omega}\sin{n\Omega t}\bigg|_{-\frac{T}{2}}^0 +\frac{4U_0}{T}\left(\frac{t}{n\Omega}\sin{n\Omega t}\bigg|_{-\frac{T}{2}}^0 \right) \right]-\int\limits_{-\frac{T}{2}}^0 \frac{1}{n\Omega}\sin{n\Omega t}dt+\frac{U_0}{n\Omega}\sin{n\Omega t}\bigg|_0^{\frac{T}{2}}-\frac{4U_0}{T}\left(\frac{t}{n\Omega}\sin{n\Omega t}\bigg|_0^{\frac{T}{2}}-\int\limits_0^\frac{T}{2}\frac{1}{n\Omega}\right)
\end{gather}
\begin{equation}
	a_0=\frac{2}{T}\left[\int\limits_{-\frac{T}{2}}^0 (U_0+\frac{4U_0}{T}t)dt + \int\limits_0^{\frac{T}{2}} (U_0-\frac{4U_0}{T}t)dt \right]=0+0=0
\end{equation}
И,наконец, разложение
\begin{equation}
	U(t)=\frac{4U_0}{\pi^2}\sum_{n=1}^{\infty}\frac{(1-(-1)^n)}{n^2}\cos{n\Omega t}
\end{equation}
\begin{figure}[tb]
	\centering
	\includegraphics[]{plot/triangle}
	\caption{Сигнал <<треугольник>> по 100 первым гармоникам}
	\label{fig:figure1}
\end{figure}

%%%%%%%%%%%%%%%%%%%%%%%%%%%%%%%%%%%%%%%%%%%%%%%%%%%%%%%%%%%%%%%%%%%%%%%%%%%%%%%
\subsubsection{Пила}
\begin{equation}
	U(t)=\left\{
	\begin{aligned}
		0, \quad &t\in [-\frac{T}{2},0]\\
		\frac{2U_0}{T}t, \quad &t\in[0, \frac{T}{2}]
	\end{aligned}
	\right.
\end{equation}
\begin{equation}
	a_n=\frac{2}{T}\left[\int\limits_{\frac{T}{2}}^0 0\cdot\cos{(n\Omega t)} dt+\frac{2U_0}{T}\int\limits_0^{\frac{T}{2}}t\cdot\cos(n\Omega t)dt \right]=...=\frac{U_0}{n^2\pi^2}\cdot(\cos(\pi n)-1)
\end{equation}
\begin{equation}
	b_n=\frac{2}{T}\left[\int\limits_{-\frac{T}{2}}^00\cdot\sin(n\Omega t)dt + \frac{2U_0}{T}\int\limits_0^{\frac{T}{2}}t\cdot\sin(n\Omega t)dt\right]=...=\frac{U_0}{n\pi}\cdot(-1)^{n+1}
\end{equation}
\begin{equation}
	a_0=\frac{2}{T}\left[\int\limits_{\frac{T}{2}}^0 0\cdot dt+\frac{2U_0}{T}\int\limits_0^{\frac{T}{2}}t \,dt \right]=...=\frac{U_0}{2}
\end{equation}
% Даня $a_0$ не посчитал, ну и пёс с ним. Зато посчитал Федя. Напишу сразу разложение
\begin{equation}
	U(t)=\frac{U_0}{4}+\sum_{n=1}^{\infty}\left[\frac{U_0}{\pi^2n^2}\left((-1)^n-1\right)\cdot\cos(n\Omega t)+ (-1)^{n+1}\frac{U_0}{\pi n}\cdot\sin(n\Omega t)\right]
\end{equation}
\begin{figure}[tb]
	\centering
	\includegraphics[]{plot/pila}
	\caption{Сигнал <<пила>> по 100 первым гармоникам}
	\label{fig:figure1}
\end{figure}

%%%%%%%%%%%%%%%%%%%%%%%%%%%%%%%%%%%%%%%%%%%%%%%%%%%%%%%%%%%%%%%%%%%%%%%%%%%%%%%
\subsubsection{Меандр}
\begin{equation}
	U(t)=\left\{
	\begin{aligned}
		-U_0,\quad &t\in [-\frac{T}{2},0]\\
		U_0,\quad &t\in[0, \frac{T}{2}]
	\end{aligned}
	\right.
\end{equation}
В силу нечетности $a_n\equiv 0$.
\begin{equation}
	b_n=\frac{4U_0}{T}\int\limits_{0}^{T/2} \sin(n\Omega t)dt=\frac{2U_0}{\pi n}(1-\cos(\pi n))
\end{equation}
И тогда
\begin{equation}
	U(t)=\sum_{n=1}^{\infty}\frac{2U_0}{\pi n}(1-(-1)^n)\cdot\sin(n\Omega t)
\end{equation}
\begin{figure}[tb]
	\centering
	\includegraphics[]{plot/meandr}
	\caption{Сигнал <<меандр>> по 100 первым гармоникам}
	\label{fig:figure1}
\end{figure}

%%%%%%%%%%%%%%%%%%%%%%%%%%%%%%%%%%%%%%%%%%%%%%%%%%%%%%%%%%%%%%%%%%%%%%%%%%%%%%%
\newpage
\subsection{Спектры сигналов с установки}
\begin{figure}[H]
	\centering
	\includegraphics[width=0.85\textwidth]{pic/sig/triangle.png}
	\caption{Сигнал <<треугольник>>}
	
\end{figure}
\begin{figure}[H]
	\centering
	\includegraphics[width=0.85\textwidth]{pic/sig/pila.png}
	\caption{Сигнал <<пила>>}
	
\end{figure}
\begin{figure}[H]
	\centering
	\includegraphics[width=0.85\textwidth]{pic/sig/meandr.png}
	\caption{Сигнал <<меандр>>}
\end{figure}

\subsection{Синтезированные сигналы}
\begin{figure}[H]
	\centering
	\includegraphics[width=0.85\textwidth]{pic/sint/triangle.png}
	\caption{Сигнал <<треугольник>>}
	
\end{figure}
\begin{figure}[H]
	\centering
	\includegraphics[width=0.85\textwidth]{pic/sint/pila.png}
	\caption{Сигнал <<пила>>}
	
\end{figure}
\begin{figure}[H]
	\centering
	\includegraphics[width=0.85\textwidth]{pic/sint/meandr.png}
	\caption{Сигнал <<меандр>>}
	
\end{figure}

\subsection{Синтез амплитудно-- и частотно-- модулированных сигналов}
\begin{figure}[H]
	\centering
	\includegraphics[width=0.85\textwidth]{pic/mod/mod2.png}
	\caption{Амплитудная модуляция с $m=0.20$}
\end{figure}
\begin{figure}[H]
	\centering
	\includegraphics[width=0.85\textwidth]{pic/mod/mod1.png}
	\caption{Амплитудная модуляция с $m=1$}
\end{figure}
\begin{figure}[H]
	\centering
	\includegraphics[width=0.85\textwidth]{pic/mod/mod3.png}
	\caption{Фазовая модуляция с  $m=0.25$}
\end{figure}
\begin{figure}[H]
	\centering
	\includegraphics[width=0.85\textwidth]{pic/mod/mod4.png}
	\caption{Фазовая модуляция с  $m=1$}
\end{figure}
\subsection{Форма сигналов при дифферинцировании и интегрировании}
\subsubsection{Интегрированные сигналы}

\begin{figure}[H]
	\centering
	\includegraphics[width=0.85\textwidth]{pic/int/triangle.png}
	\caption{Сигнал <<треугольник>>}
	
\end{figure}
\begin{figure}[H]
	\centering
	\includegraphics[width=0.85\textwidth]{pic/int/pila.png}
	\caption{Сигнал <<пила>>}
	
\end{figure}
\begin{figure}[H]
	\centering
	\includegraphics[width=0.85\textwidth]{pic/int/meandr.png}
	\caption{Сигнал <<меандр>>}
	
\end{figure}

% \paragraph{Некоторые значения $\tau=RC$}

\begin{figure}[H]
	\centering
	\includegraphics[width=0.85\textwidth]{pic/int/triangle_r_1000_c_50000.png}
	\caption{Сигнал <<треугольник>>, $R=1000$ Ом, $C=50000$ пкФ}
\end{figure}
\begin{figure}[H]
	\centering
	\includegraphics[width=0.85\textwidth]{pic/int/triangle_r_1000_c_5000.png}
	\caption{Сигнал <<треугольник>>, $R=1000$ Ом, $C=5000$ пкФ}
\end{figure}

\subsubsection{Дифференцированные сигналы}

\begin{figure}[H]
	\centering
	\includegraphics[width=0.85\textwidth]{pic/diff/triangle.png}
	\caption{Сигнал <<треугольник>>}
	
\end{figure}
\begin{figure}[H]
	\centering
	\includegraphics[width=0.85\textwidth]{pic/diff/pila.png}
	\caption{Сигнал <<пила>>}
	
\end{figure}
\begin{figure}[H]
	\centering
	\includegraphics[width=0.85\textwidth]{pic/diff/meandr.png}
	\caption{Сигнал <<меандр>>}	
\end{figure}



% \section{Заключение}

\end{document}